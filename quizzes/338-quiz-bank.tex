\documentclass[12pt, oneside]{article}
\usepackage{soto-homework}
%\usepackage{graphicx}
\usepackage{circuitikz}
% \usepackage[nohead, margin=1.0in]{geometry}

\input{solution}
% set up whether we are printing assignment or solution
% \newif\ifsolution
% \solutiontrue
% \solutionfalse

% page formatting
\pagestyle{empty}
\setlength{\parindent}{0pt}
\setlength{\parskip}{8pt}

% creates true false command with indented question
% \newcommand{\tf}[1]
% {
% \vfill
% \parbox[t]{0.25\textwidth}{\bf TRUE \hspace{0.1 in} FALSE }
% \parbox[t]{0.75\textwidth}{#1 (2pt)}
% }

% creates heading for problems
% \newcommand{\problem}[1]{{\bf Problem #1}}

% creates centered heading for sections
% \newcommand{\chead}[1]
% {\begin{center}\large\textbf{#1}\end{center}
% \hrule
% \vspace{10pt}}

% solution
% \newcommand{\solution}[1]
% {\ifsolution
% Answer: {\it #1}
% \else\fi}

% end macros

\begin{document}

{\bf Quiz 1 \hfill ENSP 338 }
\begin{center}
\begin{circuitikz}[american voltages]
\draw (0,0)
 to[V=120 V] (0,4)
 %to[short, i^>=I] (3,4)
 -- (5,4)
 to[closing switch=$S_2$] (5,2.5)
 to[R, l=$R_2$] (5,0)
 -- (0,0)

 (3,4) to[closing switch=$S_1$] (3,2.5)
 to[R, l=$R_1$] (3,0);
\end{circuitikz}
\end{center}

\problem{}
In this example, $R_1$ is a light we model as a 100 ohm resistor, and $R_2$ is
a light we model as a 50 ohm resistor.  The voltage source is a standard wall
outlet at 120 volts.

When the switch is closed, how much power is dissipated in $R_1$?

a) 288W

b) 144W

c) 1.2W

\solution{
We can use $V^2/R$ to determine this

$$ 120^2 / 100 = 144 watts $$

}


\problem{}
How much power is dissipated in $R_2$?

a) 4.8W

b) 144W

c) 288W


\solution{
$$ 120^2 / 50 = 288 watts $$
}

\problem{}
How much power is delivered by the power source?

a) 432W

b) 144W

c) 288W

\solution{
The power supplied to the two bulbs must be delivered by the power supply so
the total power is

$$ 144W + 288W = 432W $$

}

\problem{}
If electricity is purchased at 0.15 USD/kWh and switch one is closed for
6 hours and switch two is closed for 4 hours, how much does the customer
owe for this electricity?

\solution{
Using $E = P \cdot t$, for each load we get

$$ E = 144W \cdot 6 hours + 288W \cdot 4 hours = 2016 Wh = 2.02 kWh $$

$$ cost = 2.02 kWh \cdot 0.15 USD/kWh = 0.30 USD $$
}

a) 0.25 USD

b) 0.30 USD

c) 0.35 USD

\newpage
\setcounter{problem}{0}
{\bf Quiz 2 \hfill ENSP 338 \hfill Fall 2014}
\begin{center}
\begin{circuitikz}[american voltages]
\draw
% voltage source
 (0,0) to[V=120 V] (0,4)
 % fuse
 to[generic=10A Fuse] (3,4)
 % top bus
 -- (11,4)
 % switch and resistor array
 (3,4) to[closing switch] (3,2.5)
 to[R, l=$R$] (3,0)
 (5,4) to[closing switch] (5,2.5)
 to[R, l=$R$] (5,0)
 (7, 4) to[closing switch] (7, 2.5)
 to[R, l=$R$] (7,0)
 (9, 4) to[closing switch] (9, 2.5)
 to[R, l=$R$] (9,0)
 (11, 4) to[closing switch] (11, 2.5)
 to[R, l=$R$] (11,0)
 % bottom bus
 -- (0,0)
;
\end{circuitikz}
\end{center}


\problem{}
Each of these loads are small heaters with a resistance of 50 ohms.  What is the maximum number of switches that can be closed without
blowing the fuse?

a) 3

b) 4

c) 5

\solution{

We find the current from one $I = V/R = 120V/50ohm = 2.4 amps$ and then
multiply by the number of switches.  4 gives 9.6 amps and 5 gives 12
amps so we can only close 4.

}

\problem{}
What is the maximum amount of power that can be delivered by this 10 amp
circuit breaker before it blows?

a) 600 W

b) 1200 W

c) 1800 W

\solution{
$$ P = V I= 120V \cdot 10A = 1200 W $$

}

\problem{}
We model the parallel loads as a number of lamps on a 100 meter 18-gauge extension cord.
A single 18 gauge wire has a resistance of 20.95 ohms per kilometer.
When only one switch is closed, the power turned to heat in the
extension cord is closest to which?

a) 12 W

b) 24 W

c) 48 W

\solution{
100 meters of the wire has a resistance of 2.1 ohms.  The extension cord has
this resistance going there and back so the total resistance in the
wires is 4.2 ohms.

The simplest method is to assume that the current in the single load is
unchanged by additional resistance in the extension cord.

$$ P = I^2 R = 2.4^2 4.2 = 24.2W $$

We can also find the equivalent resistance for all three resistors and
find the current.

$$ I = 120 / (2.1 + 50 * 2.1) = 2.21 amps $$

Notice that the small change in resistance only changes the current by
0.2 amps.  The power we get using this number is

$$ P = 2.2^2 4.2 = 20W $$

This is still closest to 24W.

}

\problem{}
When two switches are closed, how much power is turned to heat in the
wires?

a) 24 W

b) 48 W

c) 96 W

\solution{
Since the current doubles, $I^2 R$ goes up by a factor of four.
96 W
}


%%%%%%%%%%%%%%%%%%%%%%%%%%%%%%%%%%%%%%%%%%%%%%%%%%%%%%%%%%%%%%%%%%%%%%%%%%%%%%%%
\newpage
\setcounter{problem}{0}
{\bf Quiz 3 \hfill ENSP 338 \hfill Fall 2014}

\problem{}
The voltage in an 120V RMS outlet oscillates back and forth between
peaks 60 times per second.  Which of these are closest to the peak positive and negative
voltages it reaches?

a) +120V, -120V

b. +170V, -170V

c) +120V, 0V

d. +170V, 0V

\solution{
120V is the RMS voltage so the peaks are 120V * sqrt 2 so about 170V.
}

\problem{}
We have a circuit with a 120V wall outlet connected directly to a 1.7 ohm
resistor.  Which of the graphs (on the screen) correctly displays the current
and voltage?

\includegraphics[width=4.0in]{quiz-03-2.pdf}

% put the correct voltage phase shifted?

\problem{}
\begin{tabular}{c c c c c}
       & radius & length & resistivity & resistance \\
Wire 1 & r & l & $\rho$ & $R_1$\\
Wire 2 & 2r & l & $\rho$ & $R_2$\\
\end{tabular}

We have two wires with the dimensions above.
How are the two resistances related?

a. $R_1 = 2 R_2$ \hfill b. $R_1 = 4 R_2$

c. $R_1 = R_2 / 2 $ \hfill d. $R_1 = R_2 / 4 $


\problem{}
We are evaluating the present value of an identical positive payment 10 years in the
future.  Person one has a discount rate of 5\% and person two has a
discount rate of 7\%.  Which person considers the net present value to
be larger?

a) Person one

b) Person two

c) Cannot distinguish

\solution{

$$ PV = \frac{FV}{(1+r)^{10}} $$

For person one, $1+r=1.05$ and for person two, $1+r=1.07$.  Since
$1.07^{10} > 1.05^{10}$ the present value will be less for person one.

}
%%%%%%%%%%%%%%%%%%%%%%%%%%%%%%%%%%%%%%%%%%%%%%%%%%%%%%%%%%%%%%%%%%%%%%%%%%%%%%%%
\newpage
{\bf Quiz 4 \hfill ENSP 338 \hfill Fall 2014}




\problem{}
How much electricity is used?

\problem{}
How much money is spent?

\problem{}
How much carbon is emitted?

\problem{}
Simple motor payback.  A motor costs X and has Y characteristics.  Used
Z hours per day, how much will it consume?

%%%%%%%%%%%%%%%%%%%%%%%%%%%%%%%%%%%%%%%%%%%%%%%%%%%%%%%%%%%%%%%%%%%%%%%%%%%%%%%%
\newpage
{\bf Quiz 5 \hfill ENSP 338 \hfill Fall 2014}

Lighting Stuff


%%%%%%%%%%%%%%%%%%%%%%%%%%%%%%%%%%%%%%%%%%%%%%%%%%%%%%%%%%%%%%%%%%%%%%%%%%%%%%%%
\newpage
{\bf Quiz 6 \hfill ENSP 338 \hfill Fall 2014}




%%%%%%%%%%%%%%%%%%%%%%%%%%%%%%%%%%%%%%%%%%%%%%%%%%%%%%%%%%%%%%%%%%%%%%%%%%%%%%%%
\newpage
{\bf Quiz 7 \hfill ENSP 338 \hfill Fall 2014}

Motor Stuff



%%%%%%%%%%%%%%%%%%%%%%%%%%%%%%%%%%%%%%%%%%%%%%%%%%%%%%%%%%%%%%%%%%%%%%%%%%%%%%%%
\newpage
Quiz 9

Carbon intensity and circuits

%%%%%%%%%%%%%%%%%%%%%%%%%%%%%%%%%%%%%%%%%%%%%%%%%%%%%%%%%%%%%%%%%%%%%%%%%%%%%%%%
\newpage
\section{cruft}

Exercise 3
- How much energy is consumed by a 150W load that runs for 7 hours?
- If the electricity cost is 0.15 USD per kWh, what did this load cost to run?

Exercise 4
- A refrigerator uses 1.8 kWh per day.  What is the average power draw
  in watts for this refrigerator?


% quiz 1

ohms law kirchoffs laws

% quiz 2

time value of money
compound interest

% quiz 3

loads and tariffs
power and energy

% quiz 4

AC
safety and basics

% quiz 5

power factor

% quiz 6

lighting

% quiz 7

motors and transformers




% ohms and kirchoff's law questions

%%% q

What is the combined resistance of two 10 ohm resistors placed in
series?

a) 10 ohms
b) 20 ohms
c) 5 ohms


%%%
Placed in series, resistances add so 20 ohms


%%% q
Two parallel resistors

%%%


%%% q
How many 60 watt lightbulbs can be placed on a 102V circuit before the 15 amp
breaker is triggered?

%%%

%%% q
How many joules in a kilowatt hour?

%%%

%%% q
Which of the following formulas is correct for the resistance of a wire
made of a material with resistivity $\rho$, length $l$, and area $A$?

permute the fractions

%%% a

%%% q

%%% a



%%% q
%%% a
%%% q
%%% a
%%% q
%%% a
%%% q
%%% a
%%% q
%%% a
%%% q
%%% a
%%% q
%%% a

% time value of money questions

% motor questions

% transformer questions

% AC questions

% carbon intensity questions

% tariff and economics questions

% AC/DC questions
\tf{A static magnetic field can induce a current in a wire.}
\solution{False, only changing magnetic fields, from the perspective of
the wire, can induce current.}
\vfill

\tf{Synchronous speed in an AC motor depends on the frequency of the
voltage source.}
\solution{True}
\vfill

\tf{An inverter converts AC electricity to DC electricity.}
\solution{False, rectifiers convert AC to DC.}
\vfill

\tf{An ideal diode only allows current to flow in one direction.}
\solution{True}
\vfill

\tf{The purchase cost of a motor is usually greater than the electricity
used over its lifetime.}
\solution{False, electricity often exceeds motor purchase cost by at
least a factor of 10.}
\vfill



% Lighting questions

\tf{Light sources with higher color temperatures appear more blue.}
\solution{True}
\vfill

\tf{Incandescent light bulbs use phosphors to convert ultraviolet
frequencies to visible frequencies.}
\solution{False}
\vfill

\tf{Fluorescent light bulbs use phosphors to convert infrared frequencies
to visible frequencies.}
\solution{False}
\vfill

What color of light is the human eye most sensitive to? \\
a) blue \\
b) green \\
c) red
\solution{b) green}
\vfill

\problem{}
What is the rate of carbon emissions for a kerosene lantern?  Compare
to the levelized carbon output for a solar lantern.  What is CO2eq /
klmh for kerosene vs LED?

\end{document}
