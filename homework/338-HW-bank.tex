\documentclass{article}
\usepackage{soto-homework}
\usepackage{pythontex}
\usepackage{siunitx}

% set up whether we are printing assignment or solution
\newif\ifsolution
\input{solution}

\begin{document}

\chead{ENSP 338 Homework 0}
\hrule
\vspace{10pt}

{\tiny Last modified: \today}

This is an ungraded problem set to see where your background skills are.
Please answer these questions to the best of your ability.  The results
will help me adapt my teaching to the class's background.  Please also
show enough work for me to see your thought process.


\problem{}

If $\phi$ equals 30 degrees and $a$ equals 10, what is the value of
$$ a\sin\phi\cos\phi $$

\solution{
Substitute given values and evaluate
$$ 10 \sin 30 \cos 30
= 10 \cdot 0.5 \cdot 0.866
= 4.33 $$
}

\problem{}

If $f = 60$, $t= 1$ and $\phi = 0.785$, what is the value of
$$ sin(2\pi f t + \phi) $$

\solution{
When correctly evaluated with the argument to the sin interpreted as a
radian quantity, the answer is 0.707.

If however, you interpret this as degrees, you will get the incorrect
result of 0.305.

Be sure you understand the difference between radians and degrees.
}

\problem{}

If $r_1 = 10$ and $r_2 = 20$ what is the value of
$$ \left(\frac{1}{r_1} + \frac{1}{r_2}\right)^{-1} $$

\solution{
This works out to 6.7 and is the same calculation you use for parallel
resistors.
}

\problem{}

If $r = 0.02$ and $n = 10$, what is the value of
$$ \frac{1}{(1 + r)^n} $$

\solution{
Substitute given values and evaluate
$$ \frac{1}{(1 + 0.02)^{10}}
= \frac{1}{(1.02)^{10}}
= \frac{1}{1.22}
= 0.82 $$
}



\problem{Spreadsheet}

Using your Seawolf google docs account, create a Google Docs
spreadsheet entitled exactly ``338-HW-0''.  In the first
column of the spreadsheet, create the numbers 0 through 10.  In the
second column, create a formula that multiplies each of these numbers by
two.  Share the spreadsheet with me at my address (sotod@seawolf).

\problem{Time Spent}

Please estimate the amount of time you spent on this homework.

\newpage
\setcounter{problem}{0}

\chead{ENSP 338 Homework 1}
\hrule
\vspace{10pt}

{\tiny Last modified: \today}

% assessment objectives
% - ohm's law
% - series and parallel resistances
% - electrical power delivery and consumption
% - volumetric tariffs

% add a conceptual problem?


\problem{Your Basic Toaster Oven}

\subproblem
A toaster oven has a resistance of 24$\Omega$.  At a voltage of 120V,
what is the current through it?

\subproblem
How many watts of power does it draw?

\subproblem
Draw your idea of a simple circuit model of the toaster oven and the
voltage source.

\subproblem
How much current does a toaster oven with a resistance of
12$\Omega$ draw?

\subproblem
How much power does this toaster oven draw?

\subproblem
Assuming it takes 1 minute to toast two pieces of bread using the
toaster oven with a resistance of 12$\Omega$, and electricity costs
\$0.13/kWh, what is the cost of the electricity used to toast two
pieces of bread?


\problem{Your Basic Bulb}

% move these to LED and CFL questions?

\subproblem
An incandescent light bulb is rated 40W at 120V.  What is must its
resistance be in ohms?

\subproblem
What is the current in the bulb when it is operating?

\subproblem
Repeat for an incandescent light bulb rated 60W.


\problem{Your Basic Battery}

\subproblem
A battery can deliver 100 amp-hours (Ah) at 12V.  How much stored
energy does it have, in kilowatt-hours?  In joules?

\subproblem
An alkaline battery delivers approximately 3000 mAh of current at a
voltage of 1.5 volts.  Treating the battery as an ideal voltage source,
how much energy does it deliver?

\solution{

4.5 Wh or 0.0045 kWh.
}

\subproblem
If a single alkaline battery costs \$1.00 what is the cost per kilowatt
hour for this energy?

\solution{
We divide 1/0.0045 and get 222 dollars per kWh.  This is about 2000
times the cost of grid electricity.
}

\problem{Shocking Science}

\subproblem
While fixing a light in Germany, Prof. von Meier accidentally
touched a hot wire at 220V.  What would the resistance of her shoes
(which we assume to be the most significant resistance in the
circuit) have to be so that the current through her body would not
exceed 50 milliamps (mA)?

\subproblem
Draw a circuit model of this professor between the voltage source and
ground.  Include the body and shoes.  Research and estimate the
resistances of each and include in the diagram.

\subproblem
Why is it inadvisable to perform electrical work while standing
in a puddle of water, and why are wooden ladders preferable to
aluminum?

\subproblem
Draw a circuit model of the human body to explain why electricians touch
their pinkie to the wall while touching a wire with their thumb?


\problem{Time management}

Estimate how much time you spent working on this problem set to the
nearest hour.

\newpage
\setcounter{problem}{0}
\chead{ENSP 338 Homework 2}
\hrule
\vspace{10pt}

{\tiny Last modified: \today}

The problem set you turn in should be both legible and the methods and
assumptions you are using must be clear.  I strongly suggest that
you use diagrams where appropriate and clearly state your assumptions.
Typesetting of problem sets can increase the readability.  These set
will be graded on the clarity of the method as much as the result.


\problem{Gauging the Wires}

Browse the tabulated data in the Wikipedia articles on American Wire Gauge and
Electrical Resistivities of the Elements.

% TODO: use \SI macro to take care of units?

\begin{pycode}
resistivity = 1.68e-8
length = 1000
area = 3.1415 * (2.053e-3/2)**2
rCopper = resistivity * length / area
areaStr = '\\num{{{:.3g}}}'.format(area)
rCopStr = '\\num{{{:.3g}}}'.format(rCopper)
resStr = '\\num{{{:.3g}}}'.format(resistivity)
\end{pycode}

\subproblem
From the resistivity of copper, and from the wire dimensions in the AWG
table, confirm the resistance of 12-gauge copper wire, in ohms per kilometer,
that is listed in the table.  Show your work, which will include unit
conversions.  Assume room temperature.

\solution{
$$ R = \frac{\rho l}{a} $$

The cross sectional area is
\py{areaStr}
square meters

The resistance is
$$ R = \frac{\py{resStr}\Omega \cdot \py{length}}{\py{areaStr}}
= \py{rCopStr} \Omega / km
$$
}

\subproblem
Calculate the resistance of 12 AWG aluminum wire in ohms/km at room
temperature. Why is copper preferred over aluminum conductor, even though it is
more expensive?



\problem{The curse of the extension cord}

The following appliances are found in a studio kitchen: a 600 W toaster, 3
lights @ 100 W each, and a 800W microwave oven (which you may assume to behave
like a resistor). Assume a nominal voltage of 120V ac.

\subproblem
Will a single 15A circuit be able to handle all these loads operating at the
same time?

\subproblem
What is the resistance of each load?

\subproblem
Suppose all these appliances are plugged into a power strip at the end of a
long extension cord. The cord has a resistance of 0.2 ohms. How much power is
dissipated in the cord? Do you think this might be a problem?

\subproblem
What is the voltage drop in the extension cord? If the voltage at the outlet
is exactly 120.0V, what is the voltage seen by the loads?

\subproblem
How much power does each appliance actually draw under these conditions?




\problem{Power Bill}

\begin{pycode}
energykWh = 14000
rate = 0.12
energycharge = energykWh * rate
\end{pycode}

A customer uses \py{energykWh} kWh in a 30 day month and has a peak
demand of 50 kW and an average power factor of 0.7 (70\%).  Assume an
energy rate of \py{rate} USD/kWh, a demand rate of \$30 USD/kW.

\subproblem
What is the energy charge for this month?



\solution{Energy charges are based on the electrical energy consumed
over the billing time period.
$$
\py{energykWh} \textrm{kWh} \cdot
\py{rate} USD/kWh =
\py{energycharge} \textrm{USD} $$
}

\subproblem
What is the demand charge for this month?

\solution{Demand charges are based on the highest or peak level of power
consumption over the billing period.
$$ 50 kW peak \cdot 30 USD/kW peak = 1500 USD$$
}


\subproblem
What is the customer's average power demand?

\solution{The total energy for the month should equal the average power
multiplied by the total time in one month, since energy equals power
multiplied by time.
$$ P_{avg} = \textrm{average power} = \frac{14000 kWh}{30 days\cdot 24
hr/day} = 19.4 kW $$
}



\problem{REEPS2 Problem 2.11}


\problem{}
Estimate how much time you spent working on this problem set to the
nearest hour.

\newpage
\setcounter{problem}{0}
\chead{ENSP 338 Homework 3}
\hrule
\vspace{10pt}

{\tiny Last modified: \today}

\problem{}
Name two reasons why an AC power grid with transmission voltages of several
hundred kVolts and residential service at 120 -- 240 V AC is better suited to a
national power grid than a DC system with generation at 600 Volts and
residential service at 24 Volts.

\problem{}
What is the net present value of the wiring losses in a circuit over
time?  Assume the cost of wire and the benefit in increasing it.  Where
is the point of diminishing returns?

 ENSP 338 Homework 3

Due Date:  Mon 14 Oct 2013

The problem set you turn in should be both legible and the methods and
assumptions you are using must be clear.  The point of the problem set
is for you to communicate your analysis.  I strongly suggest that you
use diagrams where appropriate and clearly state your assumptions.
Typesetting of problem sets can increase the readability.  The homework
will be graded on the clarity of the method as much as the correctness
of the result.


 Problem 1

Your Basic Transformer

A step-down transformer is connected between Phase A and neutral on a
distribution circuit that has a nominal (phase-to-phase) voltage of 21
kV.  The turns ratio of the transformer is 50:1.

State your answers to two significant figures.

a. What is the rms voltage on the secondary (customer, low-voltage)
side of the transformer? (5 pts)

b. Suppose the customer's load is 14.4 kW.  What is the current on the
primary (high-voltage) side of the transformer? (5 pts)

c. Which wires need to be thicker: those on the primary or secondary
side? (5 pts)

d. Sketch a diagram that explains how one could use the same transformer
with a different connection to also provide 120V service. (5 pts)

 Problem 2

Transformer, More Realistic

A transformer has 250 turns on the primary and 50 turns on the secondary
winding.  The transformer is 96% efficient, so that 4% of the power
entering on the primary side is converted to heat.

a. If this transformer is supplying a purely resistive load of 10 kW at
120V on the secondary side, what is the current in the primary winding?
(10 pts)

b. Explain how the answer would differ if the load still measures 10kW of
real power, but is not purely resistive. (10 pts)

 Problem 3

Draw a power triangle labeling the angle, the side of the triangle
that represents real power, the side of the triangle that represents
apparent power and the side of the triangle that represents reactive
power.

a. Assume that a given load has reactive power that is equal to 30 kVAR
and real power equal to 60 kW, calculate the power factor and show your
work and the equations that you use. (5 pts)

b. Calculate the apparent power associated with the values specified in
part a. (5 pts)

c. In part a. above, the reactive power is a positive number.  What type
of load (inductive, capacitive or resistive) produces a positive
reactive power. (5 pts)

d. What type of load can be added to the load described in a. to reduce
the reactive power? (5 pts)


 Problem 4
REEPS2 3.10

a. (5 pts)

b. (5 pts)

c. (5 pts)

 Problem 5
REEPS2 3.11

a. (5 pts)

b. (5 pts)

c. (5 pts)

 Problem 6
REEPS2 3.12 (15 pts)

 Problem 7
Please estimate the amount of time you spent on this homework. (5 pts)




\problem{Christmas Lights}

\subproblem
Consider a string of Christmas lights with 50 bulbs connected in series.
The voltage across the entire string is 120V.  What is the voltage drop across
each individual bulb?

\subproblem
If each bulb draws 4W, what is its resistance?

\subproblem
What is the current through each bulb?

\subproblem
Through the entire string?


\chead{ENSP 338 Homework 4}
\chead{Due Date:  Tue 26 Nov 2013}
\hrule
\vspace{10pt}

The problem set you turn in should be both legible and the methods and
assumptions you are using must be clear.  I strongly suggest that you
use diagrams where appropriate and clearly state your assumptions.
Typesetting of problem sets can increase the readability.  The homework
will be graded on how well you communicate your method as much as the
correctness of the result.


\problem{Power Bill}

A customer uses 14000 kWh in a 30 day month and has a peak demand of 50 kW and an
average power factor of 0.7 (70\%).  Assume an energy rate of \$0.12 USD/kWh, a
demand rate of \$30 USD/kW, and a power factor adjustment rate of
\$0.00005 USD/kWh/\%.

\subproblem
What is the energy charge for this month?

\solution{Energy charges are based on the electrical energy consumed
over the billing time period.
$$14000 kWh \cdot 0.12 USD/kWh = 1680 USD $$}

\subproblem
What is the demand charge for this month?

\solution{Demand charges are based on the highest or peak level of power
consumption over the billing period.
$$ 50 kW peak \cdot 30 USD/kW peak = 1500 USD$$
}

%\paragraph{c.} What is the power factor correction charge? (5 pts)


\subproblem
What is the customer's average power?

\solution{The total energy for the month should equal the average power
multiplied by the total time in one month, since energy equals power
multiplied by time.
$$ P_{avg} = \textrm{average power} = \frac{14000 kWh}{30 days\cdot 24
hr/day} = 19.4 kW $$
}

\section{Motor Problem}

A single-phase 120V, 2 hp motor operates at a power factor of 0.70 and
an efficiency of 82\% at full load.  It operates for a duty cycle of 12
hours in every 24-hour period.  The price of electricity is \$0.15/kWh.

\paragraph{a.} What is this motor's annual operating cost (ignoring maintenance)?
(5 pts)

\solution{
To begin, we estimate the electricity use over one year.  We then
multiply by the unit cost of electricity.
We calculate the electrical power for the motor from the delivered
mechanical power (2hp) and the efficiency.
$$ P = 2hp\cdot \ufrac{746W}{1hp} \ufrac{1 unit electrical power}{0.82
units mechanical power} = 1820 W$$
We then calculate the energy over the year.
$$ 1820W \cdot 365 days \cdot \ufrac{12 hours}{day} = 7969 kWh$$
Finally, the price paid for the electricity
$$7969 kWh \cdot 0.15 USD/kWh = 1195 \textrm{USD per year}$$
}

\paragraph{b.} What would be the savings due to an efficiency increase from 82 to
87\%? (5 pts)

\solution{
To find the cost for an 87\% efficient motore, we could repeat the
calculation above, or we could equivalently multiply by the ratio in
efficiencies.
$$ \$1195 \cdot \frac{0.82}{0.87} = \$1127 $$
The difference between these two figures is 68 USD per year.
}

\paragraph{c.} What is the current drawn by this motor at full load? (5 pts)

\solution{
The RMS current in the wire will be given by dividing the apparent power
by the voltage.  We find the apparent power by dividing the real power
by the power factor.
$$ S = \textrm{apparent power} = 1820 W \cdot \frac{1}{0.7} = 2.6 kVA $$
$$ I = S/V = 2600 VA / 120 = 21.7 A$$
If we had used the real power, we would have underestimated the current
in the wiring.
}

\paragraph{d.} Discuss the financial savings to the owner of increasing the power
factor from 0.7 to 0.8, while leaving real power unchanged. (5 pts)

\solution{
An improvement in the power factor (for equal real power) would decrease
the consumption of reactive power and current.  This would have the
benefits of lowering the current in the distribution, lowering line
losses.  It could also avoid charges from the utility for poor power
factor or for the delivery of reactive power.
}


\section{Color rendering}

Explain in your own words why colors illuminated by a fluorescent light
look different than in mid-day sunshine. (10 pts)

\solution{Our perception of the color of an object is based on the
relative powers of the different wavelengths of visible radiation
reflected by the object.  If the mixture of wavelengths illuminating the
object are different, the reflected spectrum, and thus the perceived
color of the object will be different.}

\section{Lamp Light}

A lamp produces an illuminance of 60 lux on a desk 2 meters below.

\paragraph{a.} If this lamp has a even illumination over all angles, what will the
illuminance be at 4 meters away? (3 pts)

\solution{
By the inverse square law, since the distance increases by a factor of
two the illumanance will decrease by a factor of 4 and be 15 lux.
}

\paragraph{b.} What is that illuminance in foot-candles on the desk surface? (2
pts)

\solution{
Both lux and foot-candles are units of luminous flux per unit area, so
we can convert between them with a unit conversion.
$$ 60 lux \cdot \ufrac{1 fc}{10.76 lux} = 5.58 fc$$
}

\paragraph{c.} What is the total luminous flux in lumens from this lamp, assuming it
is an ideal light bulb radiating equally in every direction? (3 pts)

\solution{
Based on the assumption of uniform illumination at all angles, we
estimate that the total luminous flux is equal to the illuminance
multiplied by the area of the sphere with a radius of 2 meters.
$$ 60 lux \cdot 4 \pi r^2 $$
$$ 60 lux \cdot 4 \cdot 3.14 \cdot 4m^2 = 3014 lumens $$
This is equivalent to two 100 watt incandescent lightbulbs.
}

\paragraph{d.} What is the luminous intensity of the source in candela? (2 pts)

\solution{
Recall that luminous intensity is the lumens per solid angle and that
there are 4$\pi$ or about 12 steradians in sphere.  To get the luminous
intensity we divide the lumens by the solid angle.
$$ 3014 / (4 * 3.14) = 240 candela$$
}

\paragraph{e.} If the upper hemisphere of the lighting fixture reflects light
downward so that the entire luminous flux is radiated evenly over only half a
sphere, but the lumens are held constant, what is the luminous intensity
in candela now? (3 pts)

\solution{
Our fixture reduces the solid angle in which light radiates from the
entire sphere to half a sphere.  This means that the solid angle reduces
from 4$\pi$ to 2$\pi$ steradians or from about 12 to about 6 steradians.
The luminous intensity increases by a factor of 2 to 480 candelas.
}


\section{Lighting Costs}

A kerosene lantern produces 40 lumens while consuming 30 milliliters
(ml) of kerosene
per hour.  An LED lightbulb outputs 900 lumens while consuming 10 watts
of electrical power.

\paragraph{a.} Assume kerosene costs 1 USD/liter.  Considering only the fuel cost,
what is the cost per kilolumen-hour (klhr) for the kerosene lantern? (5 pts)

\solution{
The light output, lumens, and the fuel use are both rates.  We need to
multiply these by a unit of time to get the unit of light energy,
kilolumen-hours.  This should be reminiscent of multiplying the power by
time to get energy.
$$ \ufrac{30 ml/hour}{40 lumens} \cdot
   \ufrac{hour}{hour}
   = 0.75 \ufrac{ml}{lumen-hour}$$
$$ \ufrac{0.75 ml}{lumen-hour} \cdot
\ufrac{1000 lumen-hour}{klhr} \cdot
\ufrac{1 USD}{1000 ml} = 0.75 USD/klhr$$
}

\paragraph{b.} Assuming electricity costs 0.13 USD/kWh, what is the cost per
kilolumen-hour for the LED lightbulb? (5 pts)

\solution{
$$ \ufrac{10 watts}{900 lumens} \cdot
   \ufrac{hour}{hour} \cdot
   \ufrac{1 kWh}{1000 Wh} \cdot
   \ufrac{0.13 USD}{kWh} \cdot
   \ufrac{1000 lumen-hour}{klhr}
   = 0.0014 USD/klhr$$
}
\paragraph{c.} What do you find noticeable about this result? (5 pts)

\solution{
We see that per unit of light, kerosene light is over 500 times more
expensive (based on the fuel cost) than the electricity used to power an
LED lightbulb.
}

%- mills, offgrid-lighting.pdf
%- efficacy of around 500 lumen-hours per liter

\section{Time Spent}

Please estimate the amount of time you spent on this homework. (5 pts)

\problem{Voltage Drop}

A 2 ohm resistor is connected to a power source 100m away, with 12 AWG copper
wire. The current is 10A.

\subproblem
Sketch a schematic model of this circuit and label each part

\subproblem
What is the voltage across the resistor?

\subproblem
What is the voltage supplied by the power source?

\problem{Extrapolating from your PG\&E bill}

Find a recent PG\&E bill for your house or a friend's.

\subproblem
Estimate the total connected load in this residence, and think
about the typical times for which appliances are used.  Does the
bill make sense?

Example: Bathroom lights, 150W * 3 hrs/day * 30 d/mo = 13.5 kWh/mo

Hair dryer, 1000W * 5 min/d * 30d/mo * 1 hr/60min = 2.5 kWh/mo

Electric toothbrush charger, 4W * 24h/d * 30d/mo = 2.8 kWh/mo

Total bathroom = 19 kWh/mo

Use a spreadsheet to carry out this estimate for your whole house.

\subproblem
From the total billed monthly electric energy (kWh) consumption,
calculate the average rate of power consumption in watts.

\subproblem
Based on things you know (U.S. population, etc), estimate the
average rate of electric power consumption in the United States, in
multiples of watts.

\subproblem
Convert the average figure from c. into kilowatt-hours (kWh) per year.

\subproblem
On the internet, see if you can verify that you got the correct
order of magnitude in your estimate in d.


\problem{Problems from REEPS2}

Solve the following problems at the end of chapter 2 in the REEPS text.

Problems:  2.6, 2.9

\end{document}
