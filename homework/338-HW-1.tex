\documentclass{article}
\usepackage{soto-homework}

% set up whether we are printing assignment or solution
\newif\ifsolution
\input{solution}

\begin{document}

\chead{ENSP 338 Homework 1}
\hrule
\vspace{10pt}

% add a conceptual problem?


\problem{Your Basic Toaster Oven}

\subproblem
A toaster oven has a resistance of 24$\Omega$.  At a voltage of 120V,
what is the current through it, and how many watts of power does it
draw?

\subproblem
Draw your idea of a simple circuit model of the toaster oven.

\subproblem
Repeat for a toaster oven with a resistance of 12$\Omega$.

\subproblem
Assuming it takes 1 minute to toast two pieces of bread using the
toaster oven with a resistance of 12$\Omega$, and electricity costs
\$0.13/kWh, what is the cost of the electricity used to toast two
pieces of bread?

\subproblem
Explain in your own words why a load with lower resistance draws
more power.


\problem{Your Basic Bulb}

% move these to LED and CFL questions?

\subproblem
An incandescent light bulb is rated 40W at 120V.  What is its
resistance in ohms, and what is the current?

\subproblem
Repeat for an incandescent light bulb rated 60W.


\problem{Your Basic Battery}

\subproblem
A battery can deliver 100 amp-hours (Ah) at 12V.  How much stored
energy does it have, in kilowatt-hours?  In joules?

\subproblem
Find a rechargeable battery. Note its charging voltage and
amp-hour capacity, and calculate the stored energy in kWh and J.
Based on your observation of how many hours it takes to charge this
battery when depleted, estimate the AC charging current in amperes and
charging power in watts.

\problem{Shocking Science}

\subproblem
While fixing a light in Germany, Prof. von Meier accidentally
touched a hot wire at 220V.  What would the resistance of her shoes
(which we assume to be the most significant resistance in the
circuit) have to be so that the current through her body would not
exceed 50 milliamps (mA)?

\subproblem
Why is it inadvisable to perform electrical work while standing
in a puddle of water, and why are wooden ladders preferable to
aluminum?

\subproblem
Why do electricians touch their pinkie to the wall while touching
a wire with their thumb?

\problem{Extrapolating from your PG\&E bill}

Find a recent PG\&E bill for your house or a friend's.

\subproblem
Estimate the total connected load in this residence, and think
about the typical times for which appliances are used.  Does the
bill make sense?

Example: Bathroom lights, 150W * 3 hrs/day * 30 d/mo = 13.5 kWh/mo

Hair dryer, 1000W * 5 min/d * 30d/mo * 1 hr/60min = 2.5 kWh/mo

Electric toothbrush charger, 4W * 24h/d * 30d/mo = 2.8 kWh/mo

Total bathroom = 19 kWh/mo

Carry out this estimate for your whole house.

\subproblem
From the total billed monthly electric energy (kWh) consumption,
calculate the average rate of power consumption in watts.

\subproblem
Based on things you know (U.S. population, etc), estimate the
average rate of electric power consumption in the United States, in
multiples of watts.

\subproblem
Convert the average figure from c. into kilowatt-hours (kWh) per year.

\subproblem
On the internet, see if you can verify that you got the correct
order of magnitude in your estimate in d.


\problem{Problems from REEPS2}

Solve the following problems at the end of chapter 2 in the REEPS text.

Problems:  2.6, 2.9

\problem{Time management}

Estimate how much time you spent working on this problem set to the
nearest hour.

\end{document}
