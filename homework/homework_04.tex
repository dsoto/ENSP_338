\documentclass{article}
\usepackage{color}
\usepackage[nohead, margin=1.5in]{geometry}
\definecolor{answer-color}{rgb}{0.1, 0.7, 0.1}

% set up whether we are printing assignment or solution
\newif\ifsolution
\solutiontrue
\solutionfalse

% creates centered heading for sections
\newcommand{\chead}[1]
{\begin{center}\large\textbf{#1}\end{center}}

% solution
\newcommand{\solution}[1]
{\ifsolution
\vspace{10pt}
{\color{answer-color} #1}
\else\fi}


\begin{document}

\chead{ENSP 338 Homework 4}
\chead{Due Date:  Tue 26 Nov 2013}
\hrule
\vspace{10pt}

The problem set you turn in should be both legible and the methods and
assumptions you are using must be clear.  I strongly suggest that you
use diagrams where appropriate and clearly state your assumptions.
Typesetting of problem sets can increase the readability.  The homework
will be graded on how well you communicate your method as much as the
correctness of the result.


\section{Real World Power Bill}

A customer uses 14000 kWh in a 30 day month and has a peak demand of 50 kW and an
average power factor of 0.7 (70\%).  Assume an energy rate of \$0.12 USD/kWh, a
demand rate of \$30 USD/kW, and a power factor adjustment rate of
\$0.00005 USD/kWh/\%.

\paragraph{a.} What is the energy charge for this month? (5 pts)

\solution{Energy charges are based on the electrical energy consumed
over the billing time period.
$$14000 kWh \cdot 0.12 USD/kWh = 1680 USD $$}

\paragraph{b.} What is the demand charge for this month? (5 pts)

\solution{Demand charges are based on the highest or peak level of power
consumption over the billing period.
$$ 50 kW peak \cdot 30 USD/kW peak = 1500 USD$$
}

\paragraph{c.} What is the power factor correction charge? (5 pts)


\paragraph{d.} What is the customer's average power? (5 pts)


\section{Motor Problem}

A single-phase 120V, 2 hp motor operates at a power factor of 0.70 and
an efficiency of 82\% at full load.  It operates for a duty cycle of 12
hours in every 24-hour period.  The price of electricity is \$0.15/kWh.

\paragraph{a.} What is this motor's annual operating cost (ignoring maintenance)?
(5 pts)

\paragraph{b.} What would be the savings due to an efficiency increase from 82 to
87\%? (5 pts)

\paragraph{c.} What is the current drawn by this motor at full load? (5 pts)

\paragraph{d.} Discuss the financial savings to the owner of increasing the power
factor from 0.7 to 0.8, while leaving real power unchanged. (5 pts)

\newpage
\section{Your Basic Transformer}

A step-down transformer is connected between Phase A and neutral on a
three-phase distribution circuit that has a nominal (phase-to-phase)
voltage of 21 kV.  The turns ratio of the transformer is 50:1.

\paragraph{a.} What is the rms voltage on the secondary (customer, low-voltage)
side of the transformer? (5 pts)

\paragraph{b.} Suppose the customer's load is 14.4 kW.  What is the current on the
primary (high-voltage) side of the transformer? (5 pts)

\paragraph{c.} Which wires need to be thicker: those on the primary or secondary
side? (5 pts)

\paragraph{d.} Sketch a diagram that explains how one could use the same transformer
with a different connection to also provide 120V service. (5 pts)


\section{Color rendering}

Explain in your own words why colors illuminated by a fluorescent light
look different than in mid-day sunshine. (10 pts)

\solution{Our perception of the color of an object is based on the
relative powers of the different wavelengths of visible radiation
reflected by the object.  If the mixture of wavelengths illuminating the
object are different, the reflected spectrum, and thus the perceived
color of the object will be different.}

\section{Lamp Light}

A lamp produces an illuminance of 60 lux on a desk 2 meters below.

\paragraph{a.} If this lamp has a even illumination over all angles, what will the
illuminance be at 4 meters away? (3 pts)

\solution{By the inverse square law, the illumanance will decrease by a
factor of 4 and be 15 lux.}

\paragraph{b.} What is that illuminance in foot-candles on the desk surface? (2
pts)

\paragraph{c.} What is the total luminous flux in lumens from this lamp, assuming it
is an ideal light bulb radiating equally in every direction? (3 pts)

\paragraph{d.} What is the luminous intensity of the source in candela? (2 pts)

\paragraph{e.} If the upper hemisphere of the lighting fixture reflects light
downward so that the entire luminous flux is radiated over only half a
sphere, but the lumens are held constant, what is the luminous intensity
in candela now? (3 pts)

\paragraph{f.} How would this change if we were using a flashlight? (2 pts)


\section{Lighting Costs}

A kerosene lantern produces 40 lumens while consuming 30 ml of kerosene
per hour.  An LED lightbulb outputs 900 lumens while consuming 10 watts
of electrical power.

\paragraph{a.} Assume kerosene costs 1 USD/liter.  Considering only the fuel cost,
what is the cost per lumen-hour for the kerosene lantern? (5 pts)

\paragraph{b.} Assuming electricity costs 0.13 USD/kWh, what is the cost per
lumen-hour for the LED lightbulb? (5 pts)

\paragraph{c.} What do you find noticeable about this result? (5 pts)

%- mills, offgrid-lighting.pdf
%- efficacy of around 500 lumen-hours per liter

\section{Time Spent}

Please estimate the amount of time you spent on this homework. (5 pts)

\end{document}
