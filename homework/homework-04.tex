\documentclass{article}
\usepackage{color}
\usepackage[nohead, margin=1.5in]{geometry}
\definecolor{answer-color}{rgb}{0.1, 0.7, 0.1}

% set up whether we are printing assignment or solution
\newif\ifsolution
\solutiontrue
%\solutionfalse

% creates centered heading for sections
\newcommand{\chead}[1]
{\begin{center}\large\textbf{#1}\end{center}}

% solution
\newcommand{\solution}[1]
{\ifsolution
\vspace{10pt}
{\color{answer-color} #1}
\else\fi}

\newcommand{\ufrac}[2]{\frac{\textrm{#1}}{\textrm{#2}}}

\begin{document}

\chead{ENSP 338 Homework 4}
\chead{Due Date:  Tue 26 Nov 2013}
\hrule
\vspace{10pt}

The problem set you turn in should be both legible and the methods and
assumptions you are using must be clear.  I strongly suggest that you
use diagrams where appropriate and clearly state your assumptions.
Typesetting of problem sets can increase the readability.  The homework
will be graded on how well you communicate your method as much as the
correctness of the result.


\section{Power Bill}

A customer uses 14000 kWh in a 30 day month and has a peak demand of 50 kW and an
average power factor of 0.7 (70\%).  Assume an energy rate of \$0.12 USD/kWh, a
demand rate of \$30 USD/kW, and a power factor adjustment rate of
\$0.00005 USD/kWh/\%.

\paragraph{a.} What is the energy charge for this month? (5 pts)

\solution{Energy charges are based on the electrical energy consumed
over the billing time period.
$$14000 kWh \cdot 0.12 USD/kWh = 1680 USD $$}

\paragraph{b.} What is the demand charge for this month? (5 pts)

\solution{Demand charges are based on the highest or peak level of power
consumption over the billing period.
$$ 50 kW peak \cdot 30 USD/kW peak = 1500 USD$$
}

%\paragraph{c.} What is the power factor correction charge? (5 pts)


\paragraph{c.} What is the customer's average power? (5 pts)

\solution{The total energy for the month should equal the average power
multiplied by the total time in one month, since energy equals power
multiplied by time.
$$ P_{avg} = \textrm{average power} = \frac{14000 kWh}{30 days\cdot 24
hr/day} = 19.4 kW $$
}

\section{Motor Problem}

A single-phase 120V, 2 hp motor operates at a power factor of 0.70 and
an efficiency of 82\% at full load.  It operates for a duty cycle of 12
hours in every 24-hour period.  The price of electricity is \$0.15/kWh.

\paragraph{a.} What is this motor's annual operating cost (ignoring maintenance)?
(5 pts)

\solution{
To begin, we estimate the electricity use over one year.  We then
multiply by the unit cost of electricity.
We calculate the electrical power for the motor from the delivered
mechanical power (2hp) and the efficiency.
$$ P = 2hp\cdot \ufrac{746W}{1hp} \ufrac{1 unit electrical power}{0.82
units mechanical power} = 1820 W$$
We then calculate the energy over the year.
$$ 1820W \cdot 365 days \cdot \ufrac{12 hours}{day} = 7969 kWh$$
Finally, the price paid for the electricity
$$7969 kWh \cdot 0.15 USD/kWh = 1195 \textrm{USD per year}$$
}

\paragraph{b.} What would be the savings due to an efficiency increase from 82 to
87\%? (5 pts)

\solution{
To find the cost for an 87\% efficient motore, we could repeat the
calculation above, or we could equivalently multiply by the ratio in
efficiencies.
$$ \$1195 \cdot \frac{0.82}{0.87} = \$1127 $$
The difference between these two figures is 68 USD per year.
}

\paragraph{c.} What is the current drawn by this motor at full load? (5 pts)

\solution{
The RMS current in the wire will be given by dividing the apparent power
by the voltage.  We find the apparent power by dividing the real power
by the power factor.
$$ S = \textrm{apparent power} = 1820 W \cdot \frac{1}{0.7} = 2.6 kVA $$
$$ I = S/V = 2600 VA / 120 = 21.7 A$$
If we had used the real power, we would have underestimated the current
in the wiring.
}

\paragraph{d.} Discuss the financial savings to the owner of increasing the power
factor from 0.7 to 0.8, while leaving real power unchanged. (5 pts)

\solution{
An improvement in the power factor (for equal real power) would decrease
the consumption of reactive power and current.  This would have the
benefits of lowering the current in the distribution, lowering line
losses.  It could also avoid charges from the utility for poor power
factor or for the delivery of reactive power.
}


\section{Color rendering}

Explain in your own words why colors illuminated by a fluorescent light
look different than in mid-day sunshine. (10 pts)

\solution{Our perception of the color of an object is based on the
relative powers of the different wavelengths of visible radiation
reflected by the object.  If the mixture of wavelengths illuminating the
object are different, the reflected spectrum, and thus the perceived
color of the object will be different.}

\section{Lamp Light}

A lamp produces an illuminance of 60 lux on a desk 2 meters below.

\paragraph{a.} If this lamp has a even illumination over all angles, what will the
illuminance be at 4 meters away? (3 pts)

\solution{
By the inverse square law, since the distance increases by a factor of
two the illumanance will decrease by a factor of 4 and be 15 lux.
}

\paragraph{b.} What is that illuminance in foot-candles on the desk surface? (2
pts)

\solution{
Both lux and foot-candles are units of luminous flux per unit area, so
we can convert between them with a unit conversion.
$$ 60 lux \cdot \ufrac{1 fc}{10.76 lux} = 5.58 fc$$
}

\paragraph{c.} What is the total luminous flux in lumens from this lamp, assuming it
is an ideal light bulb radiating equally in every direction? (3 pts)

\solution{
Based on the assumption of uniform illumination at all angles, we
estimate that the total luminous flux is equal to the illuminance
multiplied by the area of the sphere with a radius of 2 meters.
$$ 60 lux \cdot 4 \pi r^2 $$
$$ 60 lux \cdot 4 \cdot 3.14 \cdot 4m^2 = 3014 lumens $$
This is equivalent to two 100 watt incandescent lightbulbs.
}

\paragraph{d.} What is the luminous intensity of the source in candela? (2 pts)

\solution{
Recall that luminous intensity is the lumens per solid angle and that
there are 4$\pi$ or about 12 steradians in sphere.  To get the luminous
intensity we divide the lumens by the solid angle.
$$ 3014 / (4 * 3.14) = 240 candela$$
}

\paragraph{e.} If the upper hemisphere of the lighting fixture reflects light
downward so that the entire luminous flux is radiated evenly over only half a
sphere, but the lumens are held constant, what is the luminous intensity
in candela now? (3 pts)

\solution{
Our fixture reduces the solid angle in which light radiates from the
entire sphere to half a sphere.  This means that the solid angle reduces
from 4$\pi$ to 2$\pi$ steradians or from about 12 to about 6 steradians.
The luminous intensity increases by a factor of 2 to 480 candelas.
}


\section{Lighting Costs}

A kerosene lantern produces 40 lumens while consuming 30 milliliters
(ml) of kerosene
per hour.  An LED lightbulb outputs 900 lumens while consuming 10 watts
of electrical power.

\paragraph{a.} Assume kerosene costs 1 USD/liter.  Considering only the fuel cost,
what is the cost per kilolumen-hour (klhr) for the kerosene lantern? (5 pts)

\solution{
The light output, lumens, and the fuel use are both rates.  We need to
multiply these by a unit of time to get the unit of light energy,
kilolumen-hours.  This should be reminiscent of multiplying the power by
time to get energy.
$$ \ufrac{30 ml/hour}{40 lumens} \cdot
   \ufrac{hour}{hour}
   = 0.75 \ufrac{ml}{lumen-hour}$$
$$ \ufrac{0.75 ml}{lumen-hour} \cdot
\ufrac{1000 lumen-hour}{klhr} \cdot
\ufrac{1 USD}{1000 ml} = 0.75 USD/klhr$$
}

\paragraph{b.} Assuming electricity costs 0.13 USD/kWh, what is the cost per
kilolumen-hour for the LED lightbulb? (5 pts)

\solution{
$$ \ufrac{10 watts}{900 lumens} \cdot
   \ufrac{hour}{hour} \cdot
   \ufrac{1 kWh}{1000 Wh} \cdot
   \ufrac{0.13 USD}{kWh} \cdot
   \ufrac{1000 lumen-hour}{klhr}
   = 0.0014 USD/klhr$$
}
\paragraph{c.} What do you find noticeable about this result? (5 pts)

\solution{
We see that per unit of light, kerosene light is over 500 times more
expensive (based on the fuel cost) than the electricity used to power an
LED lightbulb.
}

%- mills, offgrid-lighting.pdf
%- efficacy of around 500 lumen-hours per liter

\section{Time Spent}

Please estimate the amount of time you spent on this homework. (5 pts)

\end{document}
